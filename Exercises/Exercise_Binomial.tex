\documentclass[a4paper,12pt]{article}

\usepackage{ucs}
\usepackage[utf8]{inputenc}
\usepackage{amsmath}
\usepackage{amsfonts}
\usepackage{amssymb}
\usepackage[english]{babel}
\usepackage[T1]{fontenc}
\usepackage[pdftex]{graphicx}
\usepackage[margin=2.5cm]{geometry}

\usepackage[pdftex]{hyperref}
\usepackage{url}

\title{Frequentist coverage study: intervals for a binomial parameter}
\author{}
\date{}

\begin{document}
 \maketitle
 
 We are studying the case of a binomial law:
 
 $$P(n) = B(n;N,p) = \left( ^N_n \right) p^n (1-p)^{N-n}$$
 
 This corresponds to repeating an experiment $N$ times, and counting $n$ the number of times of ``sucess'', where the probability for this success is $p$. $n$ is a random variable and follows this binomial law, of parameters $N$ and $p$. One real-life example is the estimation of an efficiency: for instance, $n$ could be the number of counts in a particle detector,
 or passing some selection, while $N$ is the total number of particles or events considered.
 
 Interval estimation of the binomial parameter $p$ is a common problem, however nontrivial because the problem is discrete. There is abundant bibliography on the subject, and we will review
 some of the most popular methods available, as well as their frequentist coverage.
 
 \section{Prelude}
 
 \begin{enumerate}
  \item Show that the mean is $E(n) = pN$ and the variance $V(n) = N p (1-p)$.
  \item What is the maximum likelihood estimate $\hat{p}$? What is its variance?
  \item Propose a simple algorithm for generating $n$ following a binomial law $B(n;N,p)$, assuming a uniform generator between 0 and 1.
  \item Draw the binomial distribution for $N=10$, $p=0.9$.
 \end{enumerate}
 
 \section{Interval estimation}
 
 For convenience, we will use the notation 
 
 $$Z_{\alpha/2} = \Phi^{-1}(1-\alpha/2) = -\Phi^{-1}(\alpha / 2)$$
 
 where
 
 $$\Phi(Z) = \frac{1}{\sqrt{2\pi}} \int_{-\infty}^Z \exp(-t^2 / 2) \text{d} t = \frac{1+\text{erf}(Z/\sqrt{2})}{2},$$
 
 so that
 
 $$Z = \sqrt{2}\, \text{erf}^{-1} (1-\alpha).$$
 
 Example: $Z_{\alpha/2} = 1$ for $1-\alpha = 0.6827$.
 
 \begin{enumerate}
  \item \textbf{Wald interval} This procedure (that we will see should be avoided) simply substitutes $\hat{p}$ for $p$ in the variance of $n$, in a Gaussian approximation. Give explicitly
  the Wald interval for any C.L.
  \item \textbf{Wilson score interval} The Wald intervals neglect the fact that $V(\hat{p})$ depends on p. E.B. Wilson proposed in 1927 to quote the largest interval $[p_1,p_2]$ such that $p_1 + Z_{\alpha/2} \sigma(p_1) < \hat{p} < p_2-Z_{\alpha/2}\sigma(p_2)$. Give the expression for such interval.
  \item \textbf{Agresti and Coull} These authors proposed to use the midpoint $\tilde{p}$ of the Wilson interval in the formula for the Wald interval, instead of the MLE $\hat{p}$. Give the corresponding interval.
  \item \textbf{Clopper-Pearson} These intervals use the Neyman construction and ensure strict frequentist coverage. The ordering rule is as follows: the lower and upper limits are constructed separately, with C.L. $(1-\alpha)/2$. 
%   \begin{enumerate}
    Write a function that builds the confidence belt in the $(n,p)$ plane. You may first build the lower limits, then the upper limits, and combine the two together. 
   Draw it for $N=10$ and with a step size in $p$ of 0.01.
%    \item The duality with hypothesis test inversion may also be used: use this method to give the C-P interval in $p$ for given $n$ and $N$. 
%    It can be done analytically using the following identity:
%    $$\sum_{i=n}^{N} \left(^N_i\right) p^i (1-p)^{N-i} = I_p (n,N-n+1)$$
%    where $I_p (\alpha,\beta)$ is the regularised incomplete Beta function:
%    $$I_p (\alpha,\beta) = \frac{\text{B}(p;\alpha,\beta)}{\text{B}(\alpha,\beta)},$$
%    defined from the incomplete Beta function,
%    $$\text{B} (p;\alpha,\beta)=\int _{0}^{p}t^{\alpha-1}\,(1-t)^{\beta-1}\text{d}t.$$
%    The incomplete Beta function is available in \texttt{scipy} as \texttt{scipy.special.betainc}.
%   \end{enumerate}
%   \item \textbf{Feldman-Cousins} The Neyman construction is used again, this time with the likelihood ratio ordering. Write this likelihood ratio and repeat confidence belt construction
%   as above.
% %   \item \textbf{Lancaster mid-P} This methods aims at solving the large overcoverage of the Clopper-Pearson intervals. In the Neyman construction, this time half of the probability of the observed $n_\text{obs}$ is added to the tail of the probability. 
%   \item \textbf{Bayesian with uniform prior} Give the posterior Bayesian probability for $p$ assuming a uniform prior, and use it to define the corresponding central interval (with equal posterior probability on both sides). It will be useful to use the Beta distribution: 
%   $$\text{Beta}(p;\alpha,\beta) = \frac{1}{B(\alpha,\beta)} p^{\alpha-1} (1-p)^{\beta-1}.$$
%   \item \textbf{Bayesian with Jeffreys' prior} Show that Jeffreys' prior for $p$ is proportional to $1/\sqrt{p(1-p)}$. Give the Bayesian posterior probability and the corresponding central interval with this prior.
 \end{enumerate}

 \subsection{Comparison of the intervals in particular cases}
 
 Compare the intervals on $p$ from the above methods, in the following cases:
 
 \begin{itemize}
 \item $N=10$, $n=10$
  \item $N=10$, $n=9$
  \item $N=10$, $n=5$
  \item $N=100$, $n=90$
 \end{itemize}

 
 \section{Study of frequentist coverage properties of various types of intervals}
 
 For each of the intervals above, study the frequentist properties of the interval estimation for $p$. In practice this is done in the following way:
 \begin{itemize}
  \item Fix the true value of the parameters $p$ and $N$;
  \item Generate $n$ according to $B(n;N,p)$;
  \item Compute the interval knowing $n$ and $N$;
  \item Check if the true value of $p$ is in this interval;
  \item Repat $N_\text{trials}$ times (e.g. 1000 times);
  \item Report the fraction of times that the true value of $p$ was inside the interval, and compare to the target confidence level (C.L.).
 \end{itemize}

 We will study 68.27\% C.L. intervals, in 3 ways:
 
 \begin{enumerate}
  \item set $N=10$ and check coverage as a function of $p$ for $0 \leq p \leq 1$
  \item make a 2-dimensional plot of coverage as a function of $p$ and $N$, for $0 \leq p \leq 1$ and $2 \leq N \leq 20$
  \item average the previous plot over all values of $p$, to get the average coverage as a function of $N$.
 \end{enumerate}

 \begin{thebibliography}{9}
  \bibitem{Cousins:2009kz}
  R.~D.~Cousins, K.~E.~Hymes and J.~Tucker,
  %``Frequentist evaluation of intervals estimated for a binomial parameter and for the ratio of Poisson means,''
  Nucl.\ Instrum.\ Meth.\ A {\bf 612} (2010) 388
  doi:10.1016/j.nima.2009.10.156
  [arXiv:0905.3831 [physics.data-an]].
  %%CITATION = doi:10.1016/j.nima.2009.10.156;%%
  %19 citations counted in INSPIRE as of 14 Nov 2018
  \bibitem{root} \texttt{TEfficiency} class documentation, \url{https://root.cern.ch/doc/master/classTEfficiency.html}.
 \end{thebibliography}


\end{document}
 
