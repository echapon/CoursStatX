\documentclass[9pt]{beamer}

\usepackage{ucs}
\usepackage[utf8x]{inputenc}
% \usepackage{beamerthemeplain}

\usepackage{amsmath}
\usepackage{amsfonts}
\usepackage{amssymb}
\usepackage[english]{babel}
\usepackage{fontenc}
% \usepackage{verbatim}
\usepackage{graphics}
 
\usepackage{textcomp}
\usepackage[absolute,overlay]{textpos}

\usepackage{wasysym}

\usepackage{slashed}
\usepackage{array}

\usetheme{CNRScolors}

\newcommand{\afac}[1]{\noindent \textcolor{red}{{\small \sc #1}}}

\usepackage{xspace}
%units
\usepackage[squaren,Gray,mediumqspace,thinspace,textstyle]{SIunits}

\usepackage[permil]{overpic}

\usepackage{textcomp}
\usepackage[absolute,overlay]{textpos}

\newcommand{\egamma}{E_{\gamma}}
\newcommand{\unitMass}{\giga\electronvolt/c^2}
\newcommand{\unitMom}{\giga\electronvolt/c}
\newcommand{\GeV}{\giga\electronvolt}
\newcommand{\TeV}{\tera\electronvolt}
\newcommand{\fb}{\femto\barn}
\newcommand{\pb}{\pico\barn}
\newcommand{\invmub}{\micro\reciprocal\barn}
\newcommand{\invpb}{\pico\reciprocal\barn}
\newcommand{\invfb}{\femto\reciprocal\barn}
\newcommand{\invnb}{\nano\reciprocal\barn}
\newcommand{\lumi}{\ensuremath{\mathcal{L}}}
\newcommand{\lumiunit}{\centi\meter\rpsquared\usk\reciprocal\second}
\newcommand{\pomeron}{\mathbb{P}}
\newcommand{\xp}{x_\pomeron}
\newcommand{\dkap}{\ensuremath{\Delta\kappa^{\gamma}}}
\newcommand{\kap}{\ensuremath{\kappa^{\gamma}}}
\newcommand{\lam}{\ensuremath{\lambda^{\gamma}}}
\newcommand{\aOw}{\ensuremath{a_0^W}}
\newcommand{\aOz}{\ensuremath{a_0^Z}}
\newcommand{\aCw}{\ensuremath{a_C^W}}
\newcommand{\aCz}{\ensuremath{a_C^Z}}
\newcommand{\aOwL}{\ensuremath{a_0^W/\Lambda^2}}
\newcommand{\aOzL}{\ensuremath{a_0^Z/\Lambda^2}}
\newcommand{\aCwL}{\ensuremath{a_C^W/\Lambda^2}}
\newcommand{\aCzL}{\ensuremath{a_C^Z/\Lambda^2}}
% \newcommand{\Lcutoff}{\ensuremath{\Lambda_{\text{cutoff}}}}
\newcommand{\Lcutoff}{\ensuremath{\Lambda}}
\newcommand{\wwgamma}{WW\gamma}
\newcommand{\missET}{\ensuremath{\not\!\!E_T}}
\newcommand{\etmiss}{\ensuremath{\not\!\!E_T}}
\newcommand{\WWgg}{\ensuremath{WW\gamma\gamma}}
\newcommand{\ggWW}{\ensuremath{WW\gamma\gamma}}
\newcommand{\dd}{\mathop{}\mathopen{}\text{d}}

\newcommand{\gpU}[1]{\ensuremath{\text{U}({#1})}}
\newcommand{\gpSU}[1]{\ensuremath{\text{SU}({#1})}}
\newcommand{\gpO}[1]{\ensuremath{\text{O}({#1})}}
\newcommand{\gpSO}[1]{\ensuremath{\text{SO}({#1})}}

\newcommand{\DO}{D\O{}\xspace}

\newcommand{\ith}{\textsuperscript{th}}
\newcommand{\st}{\textsuperscript{st}}
\newcommand{\nd}{\textsuperscript{nd}}
\newcommand{\rd}{\textsuperscript{rd}}

% \myfig[width=3cm]{label}{images/Chap1/myfig.pdf}{The capture of my figure.}
\newcommand{\myfig}[4][width=0.7\textwidth]{%
\begin{center}
 \begin{figure}
  \includegraphics[#1]{#3}
  
 \caption{\label{#2} #4}
 \end{figure}
\end{center}
}

\newcommand\FrameText[1]{%
  \begin{textblock*}{\paperwidth}(0pt,33pt)
    \raggedleft \small #1\hspace{.5em}
  \end{textblock*}}

\newcommand{\myincgr}[4]{
     \begin{overpic}[#1]{#2}
     \put(#3){#4}
     \end{overpic}}
\newcommand{\myincgrduo}[6]{
     \begin{overpic}[#1]{#2}
     \put(#3){#4}
     \put(#5){#6}
     \end{overpic}}
{\newcommand{\cmsSymbolFace}{\text}
\newcommand{\PT}{\ensuremath{p_{\text{T}}}\xspace}
\newcommand{\pt}{\ensuremath{p_{\text{T}}}\xspace}
\newcommand{\cm}{\ensuremath{\,\text{cm}}\xspace}
\newcommand{\PGm}{\ensuremath{\mu}\xspace} % muon
% \newcommand{\GeV}{\ensuremath{{\,\text{Ge\hspace{-.08em}V}}}\xspace}
\newcommand{\GeVc}{\ensuremath{{\,\text{Ge\hspace{-.08em}V\hspace{-0.16em}/\hspace{-0.08em}}c}}\xspace}
\newcommand{\GeVcc}{\ensuremath{{\,\text{Ge\hspace{-.08em}V\hspace{-0.16em}/\hspace{-0.08em}}c}^2}\xspace}
\newcommand{\abs}[1]{\ensuremath{\lvert #1 \rvert}}

% frequently used expressions
\newcommand{\ee}{\ensuremath{e^+e^-}\xspace}
\newcommand{\mumu}{\ensuremath{\mu^+\mu^-}\xspace}
\newcommand{\Mmumu}{\ensuremath{m_{\mu\mu}}\xspace}
\newcommand{\mmumu}{\ensuremath{m_{\mu\mu}}\xspace}
\newcommand{\eexp}[1]{\ensuremath{{\text e}^{#1}}\xspace}

% \newcommand{\qqbar}{\ensuremath{{\cmsSymbolFace{q}\overline{\cmsSymbolFace{q}}}}\xspace}
\newcommand{\QQbar}{\ensuremath{{\cmsSymbolFace{Q}\overline{\cmsSymbolFace{Q}}}}\xspace}
\newcommand{\Jpsi}{\ensuremath{\cmsSymbolFace{J}\hspace{-.08em}/\hspace{-.14em}\psi}\xspace}
\newcommand{\JPsi}{\ensuremath{\cmsSymbolFace{J}\hspace{-.08em}/\hspace{-.14em}\psi}\xspace}
\newcommand{\psiP}{\ensuremath{\psi\text{(2S)}}\xspace}
\newcommand{\B}{\ensuremath{\cmsSymbolFace{B}}\xspace}
\newcommand{\D}{\ensuremath{\cmsSymbolFace{D}}\xspace}

% \newcommand{\doubleRatio}{\ensuremath{\left.\left[N_{\psiP}/N_{\Jpsi}\right]_{\PbPb}\middle/\left[N_{\psiP}/N_{\Jpsi}\right]_{\pp}\right.\xspace}}
\newcommand{\doubleRatio}{\ensuremath{\left.(N_{\psiP}/N_{\Jpsi})_{\PbPb}/(N_{\psiP}/N_{\Jpsi})_{\pp}\right.\xspace}}

\newcommand{\PgU}{\ensuremath{\Upsilon}\xspace}
\newcommand{\PgUa}{\ensuremath{\Upsilon\text{(1S)}}\xspace}
\newcommand{\PgUb}{\ensuremath{\Upsilon\text{(2S)}}\xspace}
\newcommand{\PgUc}{\ensuremath{\Upsilon\text{(3S)}}\xspace}
\newcommand{\PgUbc}{\ensuremath{\Upsilon\text{(2S+3S)}}\xspace}
\newcommand{\PgUabc}{\ensuremath{\Upsilon\text{(1S,2S,3S)}}\xspace}
\newcommand{\PgUn}{\ensuremath{\Upsilon\text{(nS)}}\xspace}

\newcommand{\W}{\ensuremath{\cmsSymbolFace{W}}\xspace}
%\newcommand{\Z}{\ensuremath{\cmsSymbolFace{Z}}\xspace}
\newcommand{\ttbar}{\ensuremath{\cmsSymbolFace{t}\bar{\cmsSymbolFace{t}}}\xspace}

\newcommand{\dndy}{\ensuremath{dN/dy}\xspace}
\newcommand{\dnchdy}{\ensuremath{dN_{\text{ch}}/dy}\xspace}
\newcommand{\dndeta}{\ensuremath{dN/d\eta}\xspace}
\newcommand{\dnchdeta}{\ensuremath{dN_{\text{ch}}/d\eta}\xspace}
\newcommand{\dndpt}{\ensuremath{dN/d\pt}\xspace}
\newcommand{\dnchdpt}{\ensuremath{dN_{\text{ch}}/d\pt}\xspace}
\newcommand{\deta}{\ensuremath{\Delta\eta}\xspace}
\newcommand{\dphi}{\ensuremath{\Delta\phi}\xspace}
\newcommand{\phistar}{\ensuremath{\phi^*}\xspace}

\newcommand {\npart}  {\ensuremath{N_{\text{part}}}\xspace}
\newcommand {\ncoll}  {\ensuremath{N_{\text{coll}}}\xspace}

\newcommand{\AAA}{\ensuremath{\text{AA}}\xspace}
\newcommand{\raa}{\ensuremath{R_{\AAA}}\xspace}
\newcommand{\rpa}{\ensuremath{R_{\text{pA}}}\xspace}
\newcommand{\taa}{\ensuremath{T_{\AAA}}\xspace}
\newcommand{\rfb}{\ensuremath{R_\text{FB}}\xspace}

% references to equations, figures or tables
\newcommand{\eq}[1]{Eq.~\eqref{#1}\xspace}
\newcommand{\fig}[1]{Fig.~\ref{#1}\xspace}
\newcommand{\tab}[1]{Table~\ref{#1}\xspace}

% collision types
\newcommand{\pp}{{\ensuremath{\text{pp}}}\xspace}
\newcommand{\ppbar}{\ensuremath{\text{p}\overline{\text{p}}}\xspace}
\newcommand{\pPb}{\ensuremath{\text{p}\text{Pb}}\xspace}
\newcommand{\ppb}{\ensuremath{\text{p}\text{Pb}}\xspace}
\newcommand{\PbPb}{\ensuremath{\text{PbPb}}\xspace}
\newcommand{\pbpb}{\ensuremath{\text{PbPb}}\xspace}
\newcommand{\AuAu}{\ensuremath{\text{AuAu}}\xspace}

% center of mass energy
\newcommand{\sqrts}{\ensuremath{\sqrt{s}}\xspace}
\newcommand{\sqrtsnn}{\ensuremath{\sqrt{s_{_{\text{NN}}}}}\xspace}

%units
\newcommand{\mbinv} {\mbox{\ensuremath{\,\text{mb}^\text{$-$1}}}\xspace}
%\newcommand{\mubinv} {\mbox{\ensuremath{\,\mu\text{b}^\text{$-$1}}}\xspace}

% program name
\providecommand{\CASCADE} {{\textsc{cascade}}\xspace}
\providecommand{\HYDJET} {{\textsc{hydjet}}\xspace}


% ljpsi
\newcommand{\Lxy}{\ensuremath{L_{xy}}\xspace}
\newcommand{\Lxyz}{\ensuremath{L_{xyz}}\xspace}
% \newcommand{\ctau}{\ensuremath{c\tau^{2D}}\xspace}
% \newcommand{\ctauxyz}{\ensuremath{c\tau^{3D}}\xspace}
\newcommand{\ctau}{\ensuremath{\ell_{\Jpsi}^{2D}}\xspace}
\newcommand{\ctauxyz}{\ensuremath{\ell_{\Jpsi}^{3D}}\xspace}

% pPb 
\newcommand{\etalab}{\ensuremath{\eta_\text{lab}}\xspace}

\graphicspath{ {../figures/}{./} }

\setbeamertemplate{navigation symbols}{}

 \newcolumntype{x}[1]{%
>{\centering\hspace{0pt}}p{#1}}%
\newcommand{\tn}{\tabularnewline}

\date[Stat2]{Sept. 27, 2018}
\title{Methods of statistical analysis and simulation}
\subtitle{Cours 1}
\author[E. Chapon]{Émilien Chapon}
% \institute[(CERN)]{CERN}
% \logo{\includegraphics[height=0.6cm]{../../CMS-Color-Label.pdf}\hspace{1.05\textwidth}\includegraphics[height=0.6cm]
% {../../LogoBadge.pdf} }

\begin{document}

{
\setbeamertemplate{footline}{}
\setbeamertemplate{headline}{}
% \logo{\includegraphics[height=1.2cm]{../../CMS-Color-Label.pdf}
% \hspace{0.94\textwidth}\includegraphics[height=1.2cm]{../../LogoBadge.pdf}}

\begin{frame}
 \maketitle
 
%  \setcounter{framenumber}{0}
\end{frame}
}

\begin{frame}
 \frametitle{Outline}
 
 \tableofcontents
\end{frame}

\section{Point estimation}

\begin{frame}
 \frametitle{Parameter estimation}
 
 Let $X$ be a random variable of pdf $f(x;\theta_0)$, with $\theta_0$ unknown. We draw $N$ independent trials of $X$, $\{x_1,\dots,x_N\}$.
 
 \begin{block}{Definition: statistic}
  Any random variable $T$ that is a function of the data, $T = T(X1,\dots,X_N)$, is called a \textbf{statistic}.
 \end{block}
 
 An \textbf{estimator} is a statistic that can be used to estimate $\theta_0$. It can have the following properties:
 
 \begin{description}
  \item[unbiased]: if $\langle t_N \rangle = \theta_0$ (otherwise the bias is $\langle t_N \rangle - \theta_0 = b_N$)
  \item[convergent] or consistent: e.g. consistency in probability, $\forall \epsilon>0, \forall \eta>0, \exists N_0 / \forall N>N_0, P(|t_N - \theta_0|>\epsilon)<\eta$
  \begin{itemize}
   \item NB: The law of large numbers is equivalent to the statement that the sample mean is a consistent estimator of the parent mean.
  \end{itemize}
  \item[optimal]: if $t_N$ is unbiased and has the smallest possible variance
  \item[efficient]: if $t_N$ is unbiased and that its variance $V(t_N) \xrightarrow[N\to\infty]{} \text{minimum variance}$ (this property can be only asymptotic)
  \item[robust]: if it does not depend on a hypothesis on the pdf
 \end{description}

\end{frame}

\begin{frame}
 \frametitle{Usual methods of constructing consistent estimators}
 
 We can use the law of large numbers:
 
 $$\frac{1}{N} \sum_{i=1}^N a(X_i) \xrightarrow[N\to\infty]{} E[a(X)] = \int a(X)f(X,\theta_0)\dd X$$
 
 We will see:
 
 \begin{itemize}
  \item the moments method
  \item the maximum likelihood method
 \end{itemize}

 
\end{frame}

\begin{frame}
 \frametitle{The moments method}
 
 Let $a(X)$ such that $E[a(X)] = \int a(X) f(X;\theta) \dd X = h(\theta)$ where $h$ is known. 
 
 If $h$ is invertible, we can find the true value of $\theta$: 
 $\theta_0 = h^{-1}(E[a]) = h^{-1}\left(\int a(X) f(X;\theta) \dd X\right)$
 
 The estimator is then:
 
 $$\hat{\theta} = h^{-1} \left( N^{-1} \sum_{i=1}^N a(x_i) \right)$$
 
 NB: $\hat{\theta}$ does not directly depend on $f$, only on the $x_i$.
\end{frame}

\begin{frame}
 \frametitle{The moments method: application}
 
 \begin{block}{1D case ($\theta \in \mathbb{R}$)}
  We take simply $a(X) = X$. Then $h(\theta_0) = \bar{X} = \mu$: it is the sample mean.
 \end{block}
 
 \begin{block}{ND case: $\vec{\theta} = (\theta_1,\dots,\theta_K)$}
  We take $a_j(X) = X^j$. Then $h_j(\vec{\theta}) = \mu_j(\vec{\theta})$: $j$-ith moment of $f(X;\vec{\theta})$
 \end{block}


\end{frame}

\begin{frame}
 \frametitle{The likelihood function}
 
 \begin{block}{Likelihood function: definition}
  Given $X$ a random variable of pdf $f(X;\theta)$, and $\{X_1,\dots,X_N\}$ $N$ independent trials, then we define
  
  $$\mathcal{L}\left(\vec{X};\theta\right) = \prod_{i=1}^N f(X_i,\theta)$$
 \end{block}
 
 NB:
 
 \begin{itemize}
  \item It is a function of the parameter, given the data: it is the ``likelihood'' of a given parameter value, given some observations in data. We sometimes write
  $\mathcal{L}(\theta | X_i) = p(X_i | \theta)$.
  \item It is tempting to consider the area under $\mathcal{L}$, but $\mathcal{L}(\theta)$ is not a probability distribution function in $\theta$: \alert{the area under $\mathcal{L}$ is meaningless}.
  \item We will see that likelihood ratios are often used.
 \end{itemize}


\end{frame}

\begin{frame}
 \frametitle{The maximum likelihood method (ML)}
 
 In general the logarithm of $\mathcal{L}$ is used: $\ln \mathcal{L}\left(\vec{X};\theta\right) = \sum_{i=1}^N \ln f(X_i;\theta)$
 
 \begin{block}{Maximum likelihood estimator}
  $$\left. \frac{\partial (\ln \mathcal{L} \left(\vec{X};\theta\right)}{\partial \theta}\right|_{\hat{\theta}_\text{ML}} = \left.\frac{\partial}{\partial \theta} \left(  \sum_{i=1}^N \ln f(X_i;\theta) \right)\right|_{\hat{\theta}_\text{ML}} = 0$$
  
  $\hat{\theta}_\text{ML}$ is the maximum likelihood estimator of $\theta$.
 \end{block}

 Note: numerical methods are often designed to look for a minimum rather than a maximum. $-2 \ln \mathcal{L}$ is more commonly used.
\end{frame}


\end{document}

