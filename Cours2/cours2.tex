\documentclass[9pt]{beamer}

\usepackage{ucs}
\usepackage[utf8x]{inputenc}
% \usepackage{beamerthemeplain}

\usepackage{amsmath}
\usepackage{amsfonts}
\usepackage{amssymb}
\usepackage[english]{babel}
\usepackage{fontenc}
% \usepackage{verbatim}
\usepackage{graphics}
 
\usepackage{textcomp}
\usepackage[absolute,overlay]{textpos}

\usepackage{wasysym}

\usepackage{slashed}
\usepackage{array}

\usetheme{CNRScolors}

\input{../newcommands.tex}
\input{../custom-definitions.tex}

\graphicspath{ {../figures/}{./} }

\setbeamertemplate{navigation symbols}{}

 \newcolumntype{x}[1]{%
>{\centering\hspace{0pt}}p{#1}}%
\newcommand{\tn}{\tabularnewline}

\date[Stat2]{Sept. 27, 2018}
\title{Methods of statistical analysis and simulation}
\subtitle{Cours 1}
\author[E. Chapon]{Émilien Chapon}
% \institute[(CERN)]{CERN}
% \logo{\includegraphics[height=0.6cm]{../../CMS-Color-Label.pdf}\hspace{1.05\textwidth}\includegraphics[height=0.6cm]
% {../../LogoBadge.pdf} }

\begin{document}

{
\setbeamertemplate{footline}{}
\setbeamertemplate{headline}{}
% \logo{\includegraphics[height=1.2cm]{../../CMS-Color-Label.pdf}
% \hspace{0.94\textwidth}\includegraphics[height=1.2cm]{../../LogoBadge.pdf}}

\begin{frame}
 \maketitle
 
%  \setcounter{framenumber}{0}
\end{frame}
}

\begin{frame}
 \frametitle{Outline}
 
 \tableofcontents
\end{frame}

\section{Point estimation}

\begin{frame}
 \frametitle{Parameter estimation}
 
 Let $X$ be a random variable of pdf $f(x;\theta_0)$, with $\theta_0$ unknown. We draw $N$ independent trials of $X$, $\{x_1,\dots,x_N\}$.
 
 \begin{block}{Definition: statistic}
  Any random variable $T$ that is a function of the data, $T = T(X1,\dots,X_N)$, is called a \textbf{statistic}.
 \end{block}
 
 An \textbf{estimator} is a statistic that can be used to estimate $\theta_0$. It can have the following properties:
 
 \begin{description}
  \item[unbiased]: if $\langle t_N \rangle = \theta_0$ (otherwise the bias is $\langle t_N \rangle - \theta_0 = b_N$)
  \item[convergent] or consistent: e.g. consistency in probability, $\forall \epsilon>0, \forall \eta>0, \exists N_0 / \forall N>N_0, P(|t_N - \theta_0|>\epsilon)<\eta$
  \begin{itemize}
   \item NB: The law of large numbers is equivalent to the statement that the sample mean is a consistent estimator of the parent mean.
  \end{itemize}
  \item[optimal]: if $t_N$ is unbiased and has the smallest possible variance
  \item[efficient]: if $t_N$ is unbiased and that its variance $V(t_N) \xrightarrow[N\to\infty]{} \text{minimum variance}$ (this property can be only asymptotic)
  \item[robust]: if it does not depend on a hypothesis on the pdf
 \end{description}

\end{frame}

\begin{frame}
 \frametitle{Usual methods of constructing consistent estimators}
 
 We can use the law of large numbers:
 
 $$\frac{1}{N} \sum_{i=1}^N a(X_i) \xrightarrow[N\to\infty]{} E[a(X)] = \int a(X)f(X,\theta_0)\dd X$$
 
 We will see:
 
 \begin{itemize}
  \item the moments method
  \item the maximum likelihood method
 \end{itemize}

 
\end{frame}

\begin{frame}
 \frametitle{The moments method}
 
 Let $a(X)$ such that $E[a(X)] = \int a(X) f(X;\theta) \dd X = h(\theta)$ where $h$ is known. 
 
 If $h$ is invertible, we can find the true value of $\theta$: 
 $\theta_0 = h^{-1}(E[a]) = h^{-1}\left(\int a(X) f(X;\theta) \dd X\right)$
 
 The estimator is then:
 
 $$\hat{\theta} = h^{-1} \left( N^{-1} \sum_{i=1}^N a(x_i) \right)$$
 
 NB: $\hat{\theta}$ does not directly depend on $f$, only on the $x_i$.
\end{frame}

\begin{frame}
 \frametitle{The moments method: application}
 
 \begin{block}{1D case ($\theta \in \mathbb{R}$)}
  We take simply $a(X) = X$. Then $h(\theta_0) = \bar{X} = \mu$: it is the sample mean.
 \end{block}
 
 \begin{block}{ND case: $\vec{\theta} = (\theta_1,\dots,\theta_K)$}
  We take $a_j(X) = X^j$. Then $h_j(\vec{\theta}) = \mu_j(\vec{\theta})$: $j$-ith moment of $f(X;\vec{\theta})$
 \end{block}


\end{frame}

\begin{frame}
 \frametitle{The likelihood function}
 
 \begin{block}{Likelihood function: definition}
  Given $X$ a random variable of pdf $f(X;\theta)$, and $\{X_1,\dots,X_N\}$ $N$ independent trials, then we define
  
  $$\mathcal{L}\left(\vec{X};\theta\right) = \prod_{i=1}^N f(X_i,\theta)$$
 \end{block}
 
 NB:
 
 \begin{itemize}
  \item It is a function of the parameter, given the data: it is the ``likelihood'' of a given parameter value, given some observations in data. We sometimes write
  $\mathcal{L}(\theta | X_i) = p(X_i | \theta)$.
  \item It is tempting to consider the area under $\mathcal{L}$, but $\mathcal{L}(\theta)$ is not a probability distribution function in $\theta$: \alert{the area under $\mathcal{L}$ is meaningless}.
  \item We will see that likelihood ratios are often used.
 \end{itemize}


\end{frame}

\begin{frame}
 \frametitle{The maximum likelihood method (ML)}
 
 In general the logarithm of $\mathcal{L}$ is used: $\ln \mathcal{L}\left(\vec{X};\theta\right) = \sum_{i=1}^N \ln f(X_i;\theta)$
 
 \begin{block}{Maximum likelihood estimator}
  $$\left. \frac{\partial (\ln \mathcal{L} \left(\vec{X};\theta\right)}{\partial \theta}\right|_{\hat{\theta}_\text{ML}} = \left.\frac{\partial}{\partial \theta} \left(  \sum_{i=1}^N \ln f(X_i;\theta) \right)\right|_{\hat{\theta}_\text{ML}} = 0$$
  
  $\hat{\theta}_\text{ML}$ is the maximum likelihood estimator of $\theta$.
 \end{block}

 Note: numerical methods are often designed to look for a minimum rather than a maximum. $-2 \ln \mathcal{L}$ is more commonly used.
\end{frame}


\end{document}

